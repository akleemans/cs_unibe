\documentclass[a4paper,10pt,headlines=3.2]{scrartcl}

%F�r Windows:
%\usepackage[T1]{fontenc}		%Umlaute
%\usepackage[latin1]{inputenc}		%latin-Zeichensatz

%\usepackage[colorlinks]{hyperref}	%Hyperlinks + Verlinkung innerhalb von PDFs

\usepackage{ucs}			%Formatierung (Linux)
\usepackage{graphicx}           	%Bilder
\usepackage[ngerman]{babel}		%Deutsche Sprache
\usepackage{amsmath}			%Math. Zeichen
\usepackage{pifont}			%Skalierbare Schriftart
\usepackage{array}			%Arrays
\usepackage{epsfig}			%Erweiterte Grafiken
\usepackage{makeidx}			%Stichwortverzeichnis
\usepackage[pdftex]{color} 		%Farbige PDFs
\newcommand{\changefont}[3]{    	%Definition von Schriftarten
\fontfamily{#1} \fontseries{#2} \fontshape{#3} \selectfont}
\makeindex				%Inhaltsverzeichnis erstellen
\usepackage[automark]{scrpage2}		%scrpage
\usepackage[nosectionbib]{apacite}	%Zitieren nach APA
\usepackage{lmodern}			%Font: Latin modern
\usepackage{scrpage2}           	%KOMA-Script
\usepackage{tipa}			%Phonologische Symbole
\usepackage{qtree}			%Baumstrukturen
\usepackage{pgf}			%Rastergrafiken
\usepackage{remreset}			%Fussnoten global
\makeatletter				
\@removefromreset{footnote}{chapter}	
\makeatother 				
\setcounter{tocdepth}{3}		%Inhaltsverzeichnis bis auf Tiefe 3 ausgeben
\pagestyle{scrheadings}         	%Kopfzeilen: Seitenstil scrheadings verwenden
\changefont{cmss}{m}{n}			%Schriftart: Computer-Schrift
\usepackage{listings}			%Java-Quellcode ausgeben
\lstset{numbers=left, numberstyle=\tiny, numbersep=5pt} \lstset{language=Java} 


%Manuelle Einstellung der Seitengr�sse. Sonst automatisch, siehe unten.
%\setlength{\textheight}{24cm}
%\setlength{\textwidth}{16cm}
%\setlength{\topmargin}{-2cm}
%\setlength{\oddsidemargin}{0cm}

% Groesse des Textbereiches in der Seite
\setlength{\textwidth}{16cm}
\setlength{\textheight}{22cm}
% Kopf- und Fusszeile, Hoehe und Abstand vom Text
\setlength{\headheight}{15pt}
\setlength{\headsep}{0.8cm}

%----------- wird automatisch berechnet
% Linker Seiteneinzug
\setlength{\oddsidemargin}{2.5cm} \addtolength{\oddsidemargin}{-1in}
\setlength{\evensidemargin}{2.5cm} \addtolength{\evensidemargin}{-1in}
% Andere Groessen ausrechnen (vertikal zentrieren)
\setlength{\footskip}{\headsep}
\addtolength{\footskip}{\headheight}
\setlength{\topmargin}{\paperheight}
\addtolength{\topmargin}{-\textheight}
\addtolength{\topmargin}{-\headheight}
\addtolength{\topmargin}{-\headsep}
\addtolength{\topmargin}{-\footskip}
\addtolength{\topmargin}{-2in}
\addtolength{\topmargin}{-0.5\topmargin}
%----------- 

\setlength{\headheight}{20pt}		%Abstand zur�cksetzen f�r Kopfzeile (3 Zeilen)
\setheadsepline{.4pt}			%Separate Linie im Kopf
\clearscrheadfoot
\ihead[]{Datenstrukturen und Algorithmen \\Fr�hlingssemester 2011 \\Institut f�r angewandte Mathematik} % - links
\ohead[asdasd]{�bung 12 \\Abgabetermin 26. Mai 2011 \\Adrianus Kleemans [07-111-693]} % - linke Kopfzeile 
\cfoot[\pagemark]{\pagemark} 		%mittlere Fusszeile 

\begin{document}
\section*{Theoretische Aufgaben}
\subsection*{Aufgabe 1}
Der Algorithmus funktioniert nicht. Eine Zusammenhangskomponente beim urspr�nglichen Algorithmus, welche den Knoten mit der gr�ssten Endzeit enth�lt, hat keine Kante, die aus der Zusammenhangskomponente hinausgeht. Wird der Graph nicht transponiert und nimmt man die kleinste Endzeit, so kann gilt die obige Tatsache nicht.

\subsection*{Aufgabe 2}
Mit Kruskal: Die Kante wird als erstes ausgew�hlt und zur L�sungsmenge hinzugef�gt. \\
Mit Prim: Sobald der erste Knoten an der Kante ausgew�hlt wird (was fr�her oder sp�ter geschieht, da alle Knoten im minimalen Spannbaum vorkommen m�ssen), wird danach automatisch �ber die Kante der andere Knoten ausgew�hlt, wobei die Kante also im minimalen Spannbaum vorkommt.

\subsection*{Aufgabe 3}
Wenn \texttt{(u,v)} die leichteste Kante des Graphs ist, wird sie beim Kruskal-Algorithmus immer zuerst genommen (z.B. wie in Aufgabe 2). Dies heisst, dass eine Partitionierung der Knoten \textit{V} in disjunkte Mengen \textit{S} und \textit{V-S} auf jeden Fall bei der Kante \textit{(u,v)} geschnitten wird (da sie die leichteste ist und 'geschnitten' wird).

\subsection*{Aufgabe 4}
Sequenz der Kanten (Kruskal): \textbf{AF, FI, GH, BC, GK, BG, GJ, IE, BF, GD, GL}.

\subsection*{Aufgabe 5}
Sequenz der Kanten (Prim), Startknoten \textit{J} (zuf�llig): \textbf{JG, GH, GK, GB, BC, BF, FA, FI, IE, GL, GD}.

\subsection*{Aufgabe 6}
\begin{eqnarray}
d(a), d(b), d(c), d(d), d(e) \Leftarrow  \infty   \\
a \Rightarrow S\\
d(a) \Leftarrow 0\\
d(d) \Leftarrow 1\\
d(c) \Leftarrow 3\\
d(b) \Leftarrow 4\\
d \Rightarrow S\\
d(c) \Leftarrow 2\\
c \Rightarrow S\\
d(b) \Leftarrow 3\\
d(e) \Leftarrow 8\\
b \Rightarrow S\\
d(e) \Leftarrow 5\\
e \Rightarrow S
\end{eqnarray}
Dies ergibt folgende Minimaldistanzen: \texttt{a:0, b:3, c:2, d:1, e:5}.

\subsection*{Aufgabe 7}
Wenn im Vorraus bekannt ist, dass es einen MST mit L�nge m gibt, brauchen wir nicht zu �berpr�fen, ob es einen g�ltigen Weg gibt. Das heisst, die �berpr�fung am Ende des Algorithmus' f�llt weg.

\lstset{frame=single}
\begin{lstlisting}[caption=Aufgabe 7]{Name}
bellman-ford(G, w, s)
for i := 1 to V - 1
    for each edge (u,v) in E
        relax(u,v,w)
\end{lstlisting}

\subsection*{Aufgabe 8}
Es ist zwar eleganter, den verbleibenden Knoten auch aus der Warteschlange zu entfernen, aber dies hat keine Auswirkungen auf die korrekte Arbeitsweise des Algorithmus. \\\\
Gegenbeweis: Sei der letzte Knoten der Knoten \textit{t}. Angenommen, es gibt f�r irgendeinen anderen Knoten \textit{u} einen k�rzeren Weg, der �ber \textit{t} f�hrt. Wenn jedoch die Anweisung \texttt{EXTRACT-MIN}, so sollte dieser Weg vor den anderen gegangen werden sein, und der Knoten \textit{t} w�re nicht der letzte �brigbleibende Knoten. Dies widerspricht der Annahme, \textit{t} sei der letzte Knoten. Dies kann verallgemeinert werden; Es gibt f�r einen beliebigen Knoten, der nicht der letzte Knoten ist, keinen k�rzeren Weg �ber \textit{t}. somit wurde f�r alle Knoten der k�rzeste Pfad gefunden.

\end{document}