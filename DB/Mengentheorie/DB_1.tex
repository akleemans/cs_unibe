\documentclass[a4paper,10pt,headlines=3.2]{scrartcl}
\usepackage{graphicx}           %Bilder

%\usepackage[T1]{fontenc}        %Umlaute
%\usepackage[latin1]{inputenc}   %Windows
%\usepackage[utf8x]{inputenc}	%Linux
\usepackage{ucs}

\usepackage[ngerman]{babel}     %Deutsche Sprache
\usepackage{amsmath}            %Math. Zeichen
\usepackage{pifont}             %Skalierbare Schriftart
\usepackage{array}
\usepackage{epsfig}             %Erweiterte Grafiken
\usepackage{makeidx}            %Stichwortverzeichnis
\usepackage[pdftex]{color} 

\newcommand{\changefont}[3]{
\fontfamily{#1} \fontseries{#2} \fontshape{#3} \selectfont}

\makeindex

\usepackage[automark]{scrpage2}
\usepackage[nosectionbib]{apacite}               %Zitieren

%\usepackage[colorlinks]{hyperref}%Hyperlinks

\usepackage{lmodern}
\usepackage{scrpage2}           %KOMA-Script
\usepackage{tipa}
\usepackage{qtree}

\usepackage{remreset}			%Fussnoten global
\makeatletter
\@removefromreset{footnote}{chapter}
\makeatother 

\setcounter{tocdepth}{3}

%Kopfzeilen
\pagestyle{scrheadings}         %Seitenstil scrheadings verwenden

%\setlength{\textheight}{24cm}
%\setlength{\textwidth}{16cm}
%\setlength{\topmargin}{-2cm}
%\setlength{\oddsidemargin}{0cm}

% Groesse des Textbereiches in der Seite
\setlength{\textwidth}{16cm}
\setlength{\textheight}{22cm}
% Kopf- und Fusszeile, Hoehe und Abstand vom Text
\setlength{\headheight}{15pt}
\setlength{\headsep}{0.8cm}
% Linker Seiteneinzug
\setlength{\oddsidemargin}{2.5cm} \addtolength{\oddsidemargin}{-1in}
\setlength{\evensidemargin}{2.5cm} \addtolength{\evensidemargin}{-1in}
% Andere Groessen ausrechnen (vertikal zentrieren)
\setlength{\footskip}{\headsep}
\addtolength{\footskip}{\headheight}
\setlength{\topmargin}{\paperheight}
\addtolength{\topmargin}{-\textheight}
\addtolength{\topmargin}{-\headheight}
\addtolength{\topmargin}{-\headsep}
\addtolength{\topmargin}{-\footskip}
\addtolength{\topmargin}{-2in}
\addtolength{\topmargin}{-0.5\topmargin}

%Abstand zur�cksetzen
\setlength{\headheight}{20pt}

\usepackage{listings} 
\lstset{numbers=left, numberstyle=\tiny, numbersep=5pt} \lstset{language=Java} 

\clearscrheadfoot
%\renewcommand{\headheight}{40pt} 
\ihead[]{Datenbanken \\Fr�hlingssemester 2011 \\Institut f�r angewandte Mathematik} % - linke Kopfzeile 
\ohead[asdasd]{�bungsblatt 1 \\Abgabetermin 1. M�rz 2011 \\Adrianus Kleemans [07-111-693]} % - linke Kopfzeile 
\setheadsepline{.4pt} %Separate Linie im Kopf
\cfoot[\pagemark]{\pagemark} %- mittlere Fusszeile 

\begin{document} \changefont{cmss}{m}{n} \normalsize
\section*{Aufgabe 2}

\begin{itemize}
\item[\textbf{a)}] Zu zeigen: $A \backslash B = A \cap \bar{B}$
\begin{eqnarray}
A \backslash B = A \cap \bar{B} \\
x \in A \backslash B \\
\Leftrightarrow x \in A \textrm{ und } x \notin B \\
\Leftrightarrow x \in A \textrm{ und } x \in \bar{B} \\
\Leftrightarrow x \in A \cap \bar{B}
\end{eqnarray}

\item[\textbf{b)}] Zu zeigen: Wenn $A \cup B = A \cup C$ dann $B = C$ \\
Gegenbeispiel:
\begin{eqnarray}
A = \{a_{1}, a_{2}, a_{3}, a_{4}\} \\
B = \{a_{3}, a_{4}\} \\
C = \{a_{4}\} \\
A \cup B = A \cup C = \{a_{1}, a_{2}, a_{3}, a_{4}\} \\
\textrm{Die Anfangsgleichung stimmt, jedoch: } B \neq C \\ 
\end{eqnarray}


\item[\textbf{c)}] Zu zeigen: Es gilt $A \in B$ genau dann wenn $\bar{B} \subset \bar{A}$
\begin{eqnarray}
x \in \bar{B} \Rightarrow x \in \bar{A}\\
\textrm{Dies aber nur wenn } A \subset B\\
\textrm{Denn wenn } x \in A, \textrm{ aber } x \notin B \\
\Rightarrow \bar{B} \not\subset \bar{A}
\end{eqnarray}

\item[\textbf{d)}] Zu zeigen: $A \times B = B \times A$\\
Das kartesische Produkt ist jedoch nicht kommutativ.
Gegenbeispiel:
\begin{eqnarray}
A = \{a_{1}, a_{2}, a_{3}\} \\
B = \{b_{1}, b_{2}, b_{3}\} \\
A \times B = \{(a_{1},b_{1}),(a_{1},b_{2}), \cdots\} \\
B \times A = \{(b_{1},a_{1}),(b_{1},a_{2}), \cdots\} \\
(a_{1},b_{2}) \neq (b_{1},a_{2})
\end{eqnarray}

\item[\textbf{e)}] Zu zeigen: $\mid A \times B\mid = \mid A\mid \cdot \mid B\mid$
\begin{eqnarray}
A = \{a_{1}, a_{2}, \cdots a_{n}\}, B = \{b_{1}, b_{2}, \cdots b_{m}\} \\
A \times B = \underbrace{a_{1}b_{1}, a_{1}b_{2}, \cdots a_{2}b_{1}, a_{2}b_{2}, \cdots a_{n}b_{m}}_{n \cdot m\textrm{ Elemente}} \\
\Rightarrow \mid A \times B\mid = n \cdot m  \\
\textrm{mit } \mid A \mid = n \textrm{ und } \mid B \mid = m\\
\mid A\mid \cdot \mid B\mid = n \cdot m \\
\Rightarrow \mid A \times B\mid = \mid A\mid \cdot \mid B\mid = n \cdot m 
\end{eqnarray}

\end{itemize}

\section*{Aufgabe 3}
\lstset{frame=single}

\subsection*{a)}
\begin{lstlisting}[caption=Beispiel a)]{Name}
boolean subset(boolean[][] r, boolean [][] t)
//max = n of both rows and columns of both r and t

int m,n
for m=0 to max
	for n=0 to max
		if r[m][n] and not t[m][n] then return false
return true
\end{lstlisting}

\subsection*{b)}
\begin{lstlisting}[caption=Beispiel b)]{Name}
boolean[][] union(boolean[][] r, boolean [][] t)
//max = n of both rows and columns of both r and t

boolean arr[][]
int m,n
for m=0 to max
	for n=0 to max
		if r[m][n] or t[m][n] then arr[m][n] = true 
		else arr[m][n] = false
return arr
\end{lstlisting}

\subsection*{c)}
\begin{lstlisting}[caption=Beispiel c)]{Name}
boolean[][] compose(boolean[][] r, boolean [][] t)
//max = n of both rows and columns of both r and t

boolean arr[][]
int l,m,n
for m=0 to max
   for n=0 to max
      if r[m][n] then
         for l=0 to max
            if t[n][l] then arr[m][l] = true
return arr
\end{lstlisting}

\end{document}
