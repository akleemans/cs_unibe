\documentclass[a4paper,10pt,headlines=3.2]{scrartcl}
\usepackage{graphicx}           %Bilder

%\usepackage[T1]{fontenc}        %Umlaute
%\usepackage[latin1]{inputenc}   %Windows
%\usepackage[utf8x]{inputenc}	%Linux
\usepackage{ucs}

\usepackage[ngerman]{babel}     %Deutsche Sprache
\usepackage{amsmath}            %Math. Zeichen
\usepackage{pifont}             %Skalierbare Schriftart
\usepackage{array}
\usepackage{epsfig}             %Erweiterte Grafiken
\usepackage{makeidx}            %Stichwortverzeichnis
\usepackage[pdftex]{color} 

\newcommand{\changefont}[3]{
\fontfamily{#1} \fontseries{#2} \fontshape{#3} \selectfont}

\makeindex

\usepackage[automark]{scrpage2}
\usepackage[nosectionbib]{apacite}               %Zitieren

%\usepackage[colorlinks]{hyperref}%Hyperlinks

\usepackage{lmodern}
\usepackage{scrpage2}           %KOMA-Script
\usepackage{tipa}
\usepackage{qtree}
\usepackage{wasysym}

\usepackage{remreset}			%Fussnoten global
\makeatletter
\@removefromreset{footnote}{chapter}
\makeatother 

\setcounter{tocdepth}{3}

%Kopfzeilen
\pagestyle{scrheadings}         %Seitenstil scrheadings verwenden

%\setlength{\textheight}{24cm}
%\setlength{\textwidth}{16cm}
%\setlength{\topmargin}{-2cm}
%\setlength{\oddsidemargin}{0cm}

% Groesse des Textbereiches in der Seite
\setlength{\textwidth}{16cm}
\setlength{\textheight}{22cm}
% Kopf- und Fusszeile, Hoehe und Abstand vom Text
\setlength{\headheight}{15pt}
\setlength{\headsep}{0.8cm}
% Linker Seiteneinzug
\setlength{\oddsidemargin}{2.5cm} \addtolength{\oddsidemargin}{-1in}
\setlength{\evensidemargin}{2.5cm} \addtolength{\evensidemargin}{-1in}
% Andere Groessen ausrechnen (vertikal zentrieren)
\setlength{\footskip}{\headsep}
\addtolength{\footskip}{\headheight}
\setlength{\topmargin}{\paperheight}
\addtolength{\topmargin}{-\textheight}
\addtolength{\topmargin}{-\headheight}
\addtolength{\topmargin}{-\headsep}
\addtolength{\topmargin}{-\footskip}
\addtolength{\topmargin}{-2in}
\addtolength{\topmargin}{-0.5\topmargin}

%Abstand zur�cksetzen
\setlength{\headheight}{20pt}

\usepackage{listings} 
\lstset{numbers=left, numberstyle=\tiny, numbersep=5pt} \lstset{language=Java} 
\changefont{cmss}{m}{n}

\clearscrheadfoot
%\renewcommand{\headheight}{40pt} 
\ihead[]{Datenbanken \\Fr�hlingssemester 2011 \\Institut f�r angewandte
Mathematik} % - linke Kopfzeile 
\ohead[asdasd]{�bung 4, Abgabe 22. M�rz 2011 \\Adrianus Kleemans
[07-111-693]\\Pinar Kayalar [10-123-453]} % - linke Kopfzeile 
\setheadsepline{.4pt} %Separate Linie im Kopf
\cfoot[\pagemark]{\pagemark} %- mittlere Fusszeile 


\begin{document}
\section*{Aufgabe 1}
Aus Platzgr�nden werden die Tabellen (e=employee, w=works, c=company, m=manages) und Attribute (pn=person-name, s=street, c=city, cn=company-name, ...) abgek�rzt.
\begin{enumerate}
 \item $\pi_{e.pn}(\sigma_{m.s=e.s, m.c=e.c, m.pn=e.pn}(e \times (\pi_{e.s, e.c, m.pn, m.mn}(\sigma_{e.pn=m.mn}(e \times m)))))$
 \item $\pi_{w.pn}(w-\sigma_{w.cn=``FBC''}(w))$
 \item $\pi_{R.name}(\sigma_{R.salary<S.salary}(\rho_{R}(e \Bowtie w) \times \rho_{S}(\sigma_{cn=``FBC''}(e \Bowtie w))))$
 \item $\pi_{w.pn}(w-\sigma_{w.cn=``FBC''}(w)) - \pi_{R.name}(\sigma_{R.salary\leq S.salary}(\rho_{R}(e \Bowtie w) \times \rho_{S}(\sigma_{cn=``FBC''}(e \Bowtie w))))$
 \item $\pi_{cn}(\vartheta_{\textbf{max}(pn)}(_{cn}\vartheta_{\textbf{count}(pn)}(w)))$
 \item $\pi_{cn}(\vartheta_{\textbf{min}(s)}(_{cn}\vartheta_{\textbf{sum}(s)}(w)))$
 \item $\pi_{cn}(\sigma_{R.\textbf{avg}(s)>S.\textbf{avg}(s)}(\rho_{R}(_{cn}\vartheta_{\textbf{avg}(s)}(w)) \times \rho_{S}(\sigma_{cn=``FBC''}(_{cn}\vartheta_{\textbf{avg}(s)}(w)))))$
 \item $\pi_{R.cn}(_{cn}\vartheta_{\textbf{count}(s)}(\sigma_{R.cn=S.cn, R.c=S.c}(\vartheta_{R}(c) \times \vartheta_{S}(c))))$
\end{enumerate}

\section*{Aufgabe 2}
\lstset{frame=single}

\begin{lstlisting}[caption=Aufgabe 2]{Name}
// Gibt das Tupel an Stelle i zur�ck
list get(list l, var i)
while i > 0
	l <= tail(l)
	i <= i - 1
return head(l)

// Gibt zur�ck, ob Liste leer ist
bool empty(list l)
if head(l) = NULL
	return true
return false

// Gibt Gr�sse der Liste zur�ck
var size(list l)
int s <= 0
while empty(l) = false
	s <= s + 1
	l <= tail(l)
return s
	
// F�gt zwei listen zusammen
list concatenate(tuple x, tuple y)
for var a <= size(x) to a >= 0, a <= a - 1
	cons(get(x, a), y)
return y

// Ergibt das kartesische Produkt r x s
list cartesianProduct(list r, list s)
list t
for var a <= size(r) - 1 to a > = 0, a <= a - 1)
	for var b <= size(s) to b > = 0, b <= b - 1
		cons(concatenate(get(r, a), get(s, b)), t)
return t
\end{lstlisting}

\section*{Aufgabe 3}
\begin{enumerate}
 \item F�r ein beliebiges Tupel $a_{n}$ mit Attributen $A, B, C$ gilt: Sollte es im Fall $x$ zuerst durch eine Projektion auf die Spalten A und B reduziert werden, wird danach jedes Element, f�r das $A>10$ gilt, zur�ckgeliefert. Bei $y$ werden im ersten Schritt exakt die selben Tupel ausgew�hlt, n�mlich nur die, f�r welche auch $A>10$ gilt, und davon werden dann die zwei Attribute A und B zur�ckgegeben. Die Beiden Operationen Projektion und Auswahl (\texttt{select}) k�nnen in diesem Fall ausgetauscht werden, die beiden Ergebnisse sind �quivalent.
 \item Hier sind $x$ und $y$ nicht gleichbedeutend. Wenn ein Tupel $a_{1}$ einen Wert in r.A mit einem anderen Tupel $a_{2}$ in t.A gemeinsam hat, abet r.B (in $a_{1}$) nicht mit t.B (in $a_{2}$) �bereinstimmt, wird $a_{1}$ in der Auswertung $\pi_{A}(r - t)$ erscheinen, nicht jedoch in $\pi_{A}(r) - \pi_{A}(t)$. Die Differenz wird hier nicht aus den selben Mengen gebildet.
\end{enumerate}


\end{document}