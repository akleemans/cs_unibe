\documentclass[a4paper,10pt,headlines=3.2]{scrartcl}


% Basispakete
\usepackage[utf8]{inputenc}
\usepackage{amsmath}
\usepackage[automark]{scrpage2}
\usepackage{graphicx}
\usepackage{amssymb}
\usepackage{tikz}
\usepackage{ifthen}
\usepackage{calc}
\usepackage{tikz}
\usepackage{ngerman}

\usepackage{graphicx}           %Bilder
\usepackage{amsmath}            %Math. Zeichen
\usepackage{pifont}             %Skalierbare Schriftart
\usepackage{array}
\usepackage{epsfig}             %Erweiterte Grafiken
\usepackage{makeidx}            %Stichwortverzeichnis
%\usepackage[pdftex]{color} 

\newcommand{\changefont}[3]{
\fontfamily{#1} \fontseries{#2} \fontshape{#3} \selectfont}

\makeindex

\usepackage[automark]{scrpage2}
\usepackage[nosectionbib]{apacite}               %Zitieren

%\usepackage[colorlinks]{hyperref}%Hyperlinks

\usepackage{lmodern}
\usepackage{scrpage2}           %KOMA-Script
\usepackage{tipa}
\usepackage{qtree}

\usepackage{remreset}			%Fussnoten global
\makeatletter
\@removefromreset{footnote}{chapter}
\makeatother 

\setcounter{tocdepth}{3}

%Kopfzeilen
\pagestyle{scrheadings}         %Seitenstil scrheadings verwenden

%\setlength{\textheight}{24cm}
%\setlength{\textwidth}{16cm}
%\setlength{\topmargin}{-2cm}
%\setlength{\oddsidemargin}{0cm}

% Groesse des Textbereiches in der Seite
\setlength{\textwidth}{16cm}
\setlength{\textheight}{21.3cm}
% Kopf- und Fusszeile, Hoehe und Abstand vom Text
\setlength{\headheight}{15pt}
\setlength{\headsep}{0.8cm}
% Linker Seiteneinzug
\setlength{\oddsidemargin}{2.5cm} \addtolength{\oddsidemargin}{-1in}
\setlength{\evensidemargin}{2.5cm} \addtolength{\evensidemargin}{-1in}
% Andere Groessen ausrechnen (vertikal zentrieren)
\setlength{\footskip}{\headsep}
\addtolength{\footskip}{\headheight}
\setlength{\topmargin}{\paperheight}
\addtolength{\topmargin}{-\textheight}
\addtolength{\topmargin}{-\headheight}
\addtolength{\topmargin}{-\headsep}
\addtolength{\topmargin}{-\footskip}
\addtolength{\topmargin}{-2in}
\addtolength{\topmargin}{-0.5\topmargin}

%Abstand zurücksetzen
\setlength{\headheight}{20pt}

\usepackage{listings} 
\lstset{numbers=left, numberstyle=\tiny, numbersep=5pt} \lstset{language=Java} 
\changefont{cmss}{m}{n}

%Tabellen
\usepackage{booktabs}
\usepackage{multirow}

%Kopfzeilen
\clearscrheadfoot
%\renewcommand{\headheight}{40pt} 
\ihead[]{Rechnerarchitektur \\Frühlingssemester 2012 \\Institut für angewandte
Mathematik} % - linke Kopfzeile 
\ohead[asdasd]{Übung 1, Abgabe 15. März 2011 \\Adrianus Kleemans
[07-111-693] \\Christa Biberstein [09-104-381]} % - linke Kopfzeile 
\setheadsepline{.4pt} %Separate Linie im Kopf
\cfoot[\pagemark]{\pagemark} %- mittlere Fusszeile 

% Nummerierung
\renewcommand \thesection {\arabic{section}.}
\renewcommand \thesubsection {\arabic{section}.\arabic{subsection}.}
\renewcommand \thesubsubsection {\arabic{section}.\arabic{subsection}.\arabic{subsubsection}.}

\begin{document}

%\chapter{Theorieteil}
\section{Aufgabe}
Der String \texttt{Test} ist 5 Bytes lang (4 Zeichen + ``$\backslash$0'')

\section{Aufgabe}
int:
\begin{verbatim}
int getAt(int *a, int i) {
  return *(a+i);
}
\end{verbatim}
short:
\begin{verbatim}
short getAt(short *a, int i) {
  return *(a+i);
}
\end{verbatim}
Die Berechnung des $i$-ten Elements eines Arrays $a$ erfolgt ``dynamisch'', nach Speichertyp,  d.h. bei einem double wird bei \texttt{*(a+i)} für $i$ in 8-Byte Schritten gesprungen, bei int/short-Typen um 4 Byte (je nach Systemarchitektur, vgl. \texttt{sizeof()}).

\section{Aufgabe}
\begin{itemize}
 \item Zeile 2: \texttt{bffff844} (Adresse von \texttt{b})
 \item Zeile 3: \texttt{3ade68b1} (Wert von \texttt{b} als \texttt{long})
 \item Zeile 4: \texttt{68} (p wurde um 1 im Speicher ``verschoben'', d.h. zeigt nun auf \texttt{bffff845})
 \item Zeile 5: \texttt{de} (p ist nun \texttt{bffff46})
 \item Zeile 6: \texttt{bffff47}
\end{itemize}

\section{Aufgabe}
\begin{itemize}
 \item Programm 1: Bei \texttt{increment} wurde die Adresse von i übergeben. Beide Variablen haben nach Programmduchlauf den Wert \texttt{1338}.
 \item Es wird 1337 zurückgegeben, also \texttt{j = 1337}. Danach wird an der Speicheradresse eins addiert, d.h. \texttt{i = 1338}.
\end{itemize}

\section{Aufgabe}
Programmstück 1:
\begin{itemize}
 \item Zeile 3: Ausgabe: \texttt{1 1}. Es wird zwei Mal das erste Element von \texttt{x} ausgegeben.
 \item Zeile 5: Ausgabe: \texttt{1 2}. Der Pointer \texttt{px} zeigt seit Zeile 4 auf das zweite Element.
\end{itemize}
\noindent Programmstück 2:
\begin{itemize}
 \item Zeile 5: Ausgabe: \texttt{10 11}. Es wird der Speicherbereich an der Adresse vor \texttt{x} überschrieben, welcher vorher nicht für das Programm reserviert wurde. Falls dort andere Variablen oder geschützter Speicher ist, wird versucht, diesen zu überschreiben, was unter Umständen zu Fehlern führen kann.
\end{itemize}

\section{Aufgabe}
\begin{itemize}
 \item Zeile 14: \texttt{8}: Alle Elemente zusammengezählt (\texttt{4 + 1 + 1 + 2}) ergibt eine Speicherbelegung von 8 Byte.
 \item Zeile 15: \texttt{8}: Das grösste Element der Union ist der \texttt{char}, welcher 8 Byte belegt.
\end{itemize}

\begin{center}
\begin{tabular}{|l|l|l|}
\hline
\multicolumn{1}{|c|}{MyStruct} \\
\hline
\multirow{2}{*}{\textbf{short int} d}   \\
 \\ \hline
\textbf{char} c \\ \hline
\textbf{char} b \\ \hline
\multirow{4}{*}{\textbf{char} a[4]}
  \\ 
  \\ 
  \\ 
  \\ \hline
\end{tabular}
\end{center}
\textbf{MyStruct}: Die Elemente werden nacheinander im Speicher platziert. Die Gesamtgrösse setzt sich aus den addierten Grössen der einzelnen Variablen zusammen.

\begin{center}
\begin{tabular}{|l|l|l|}
\hline
\multicolumn{3}{|c|}{MyUnion} \\
\hline
 & \multirow{8}{*}{\textbf{char} a[8]} & \multirow{8}{*}{\textbf{short int} d[4]} \\ 
 & & \\ 
 & & \\ 
 & & \\ \cline{1-1}
\multirow{4}{*}{\textbf{int} b} & & \\ 
 & & \\ 
 & & \\ 
 & & \\ \hline
\end{tabular}
\end{center}
\textbf{MyUnion}: Die Elemente belegen den gleichen Speicherplatz. Das grösste Element bestimmt den Speicherplatz der \texttt{union}.

\section{Aufgabe}
Mit \texttt{\#define} wird eine vorgegebene Zeichenfolge durch eine andere ersetzt. Hier wird \texttt{callA} mit \texttt{callB(1)} ersetzt, was zum Aufruf der Funktion \texttt{callB} führt, mit Parameter \texttt{1}. Dies hat den Output $(1+2=)$ \texttt{3} zur Folge.

\end{document}
