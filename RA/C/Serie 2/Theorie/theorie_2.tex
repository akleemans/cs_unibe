\documentclass[a4paper,10pt,headlines=3.2]{scrartcl}


% Basispakete
\usepackage[utf8]{inputenc}
\usepackage{amsmath}
\usepackage[automark]{scrpage2}
\usepackage{graphicx}
\usepackage{amssymb}
\usepackage{tikz}
\usepackage{ifthen}
\usepackage{calc}
\usepackage{tikz}
\usepackage{ngerman}

\usepackage{graphicx}           %Bilder
\usepackage{amsmath}            %Math. Zeichen
\usepackage{pifont}             %Skalierbare Schriftart
\usepackage{array}
\usepackage{epsfig}             %Erweiterte Grafiken
\usepackage{makeidx}            %Stichwortverzeichnis
%\usepackage[pdftex]{color} 

\newcommand{\changefont}[3]{
\fontfamily{#1} \fontseries{#2} \fontshape{#3} \selectfont}

\makeindex

\usepackage[automark]{scrpage2}
\usepackage[nosectionbib]{apacite}               %Zitieren

%\usepackage[colorlinks]{hyperref}%Hyperlinks

\usepackage{lmodern}
\usepackage{scrpage2}           %KOMA-Script
\usepackage{tipa}
\usepackage{qtree}

\usepackage{remreset}			%Fussnoten global
\makeatletter
\@removefromreset{footnote}{chapter}
\makeatother 

\setcounter{tocdepth}{3}

%Kopfzeilen
\pagestyle{scrheadings}         %Seitenstil scrheadings verwenden

%\setlength{\textheight}{24cm}
%\setlength{\textwidth}{16cm}
%\setlength{\topmargin}{-2cm}
%\setlength{\oddsidemargin}{0cm}

% Groesse des Textbereiches in der Seite
\setlength{\textwidth}{16cm}
\setlength{\textheight}{21.3cm}
% Kopf- und Fusszeile, Hoehe und Abstand vom Text
\setlength{\headheight}{15pt}
\setlength{\headsep}{0.8cm}
% Linker Seiteneinzug
\setlength{\oddsidemargin}{2.5cm} \addtolength{\oddsidemargin}{-1in}
\setlength{\evensidemargin}{2.5cm} \addtolength{\evensidemargin}{-1in}
% Andere Groessen ausrechnen (vertikal zentrieren)
\setlength{\footskip}{\headsep}
\addtolength{\footskip}{\headheight}
\setlength{\topmargin}{\paperheight}
\addtolength{\topmargin}{-\textheight}
\addtolength{\topmargin}{-\headheight}
\addtolength{\topmargin}{-\headsep}
\addtolength{\topmargin}{-\footskip}
\addtolength{\topmargin}{-2in}
\addtolength{\topmargin}{-0.5\topmargin}

%Abstand zurücksetzen
\setlength{\headheight}{20pt}

\usepackage{listings} 
\lstset{numbers=left, numberstyle=\tiny, numbersep=5pt} \lstset{language=Java} 
\changefont{cmss}{m}{n}

%Tabellen
\usepackage{booktabs}
\usepackage{multirow}

%Kopfzeilen
\clearscrheadfoot
%\renewcommand{\headheight}{40pt} 
\ihead[]{Rechnerarchitektur \\Frühlingssemester 2012 \\Institut für angewandte
Mathematik} % - linke Kopfzeile 
\ohead[asdasd]{Übung 2, Abgabe 29. März 2011 \\Adrianus Kleemans
[07-111-693] \\Christa Biberstein [09-104-381]} % - linke Kopfzeile 
\setheadsepline{.4pt} %Separate Linie im Kopf
\cfoot[\pagemark]{\pagemark} %- mittlere Fusszeile 

% Nummerierung
\renewcommand \thesection {\arabic{section}.}
\renewcommand \thesubsection {\arabic{section}.\arabic{subsection}.}
\renewcommand \thesubsubsection {\arabic{section}.\arabic{subsection}.\arabic{subsubsection}.}

\begin{document}
\section{Function Pointer}
\begin{verbatim}
A: 10
B: 11
C: 12
\end{verbatim}

\section{Const}
\begin{itemize}
\item \texttt{int * a}: Zeiger auf int
\item \texttt{int const * b}: Zeiger auf konstanten int
\item \texttt{int * const c}: Konstanter Zeiger auf int
\item \texttt{int const * const d}: Konstanten Zeiger auf konstanten int
\end{itemize}


\section{Arrays}
Korrekt müsste Zeile 3 heissen: \texttt{for (i=0; i<10; i++) \{}.\\
Der Array hat 10 Elemente: Das Problem mit der Schleife der Aufgabe ist, dass bei der letzten Iteration auf Element \texttt{array[10]} zugegriffen wird, welches ein ungültiger Zugriff ist (der Array wurde nur von \texttt{array[0]} bis \texttt{array[9]} deklariert).

\section{MIPS Branch Instructions}
\begin{verbatim}
int i;
for (i = 11; i != 0; i--) {
  #do something
}
\end{verbatim}

\section{MIPS More Branch Instructions}
\begin{verbatim}
slt $at, $s2, $s1
beq $at, $zero, Label
\end{verbatim}
Zuerst wird, wenn, wenn $s2$ (``set on less than'') kleiner ist als $s1$, $at$ auf 1 gesetzt. Dann wird per $beq$ (``branch on equal'') geprüft, ob $at$ 0 ist, was in jedem anderen Fall zutrifft (also wenn $s1 \geq s2$).

\section{Endianness}
\begin{center}
\begin{tabular}{ccccc}
\textbf{Speicheradresse} & \textbf{10004} & \textbf{10005} & \textbf{10006} & \textbf{10007}\\
\textbf{Wert (hex)} & 6d & 8c & 24 & 00
\end{tabular}
\end{center}

\begin{itemize}
\item[(a)] \textbf{Big Endian} (\textit{most significant byte} an der niedrigsten Speicheradresse): \\
\texttt{10007} ist das niederwertigste Byte\\
Wert: $6d8c2400_{16} = 1837900800_{10}$
\item[(b)] \textbf{Little Endian} (\textit{least significant byte} an der niedrigsten Speicheradresse): \\
\texttt{10004} ist das niederwertigste Byte\\
Wert: $00248c6d_{16} = 2395245_{10}$
\end{itemize}

\section{Array-Zugriff}
C-Programm:
\begin{verbatim}
void mul(short x[], int index, int mul) {
  x[index] *= mul;
}
\end{verbatim}
Adresse des ersten Arrayelements: \$t2 \\
\texttt{index}: \$t1 \\
\texttt{mul}: \$t0 \\\\
Assembler:
\begin{verbatim}
mul: add $t3, $t1, $t1  // index*2 berechnen (da short-Array)
     add $t4, $t2, $t3  // ...und damit Adresse von x[index] berechnen
                        // $t4 enthält jetzt genaue Adresse)
     
     mult $t4, $t0      // multiplizieren
     mflo $t4           // Resultat holen und unter $t4 speichern
     
     jr $ra             // zurückspringen (temporäre $tX-Register müssen nicht
                        // restored werden)
\end{verbatim}


\section{MIPS Register/Hauptspeicher}
\begin{verbatim}
lw $s2, 40($s1)
\end{verbatim}
Angenommen, der Array ist ein Integer-Array, multiplizieren wir den Speicherbedarf der einzelnen Werte des Arrays (4 Byte) mit der Anzahl zu überspringenden Wörter (10) \texttt{= 40}.
\end{document}
